\documentclass[a4paper]{article}

\usepackage{ragged2e} % патек для нормального выравнивания текста

\usepackage{titlesec} % пакет для работы с разделами 
\newcommand{\sectionbreak}{\clearpage} % что каждый раздел начинался с новой страницы

\usepackage{ragged2e} % патек для нормального выравнивания текста

\usepackage{geometry} % пакет для работы с разметкой страницы
\geometry{left=2.5cm} % левый отступ
\geometry{right=1.0cm} % правый отступ
\geometry{top=0.5cm} % верхний отступ
\geometry{bottom=0.5cm} % нижний отступ
\geometry{textwidth=175mm}
\geometry{textheight=277mm}

% настройка шрифтов и языков
    \usepackage[english, russian]{babel} % для работы с английским и русским языками
    \usepackage[T2A]{fontenc} % для норм паказа символов в pdf
    \usepackage[utf8]{inputenc} % для норм взаимодействия символов для самого tex

    \usepackage{fontspec} % для работы со шрифтами 
    \usepackage{polyglossia} % Поддержка многоязычности
    \defaultfontfeatures{Ligatures=TeX,Mapping=tex-text}

    \setmainlanguage[babelshorthands = true]{russian} % устанавливаемя русский как основной язык
    \setotherlanguage{english} % устанавливаем английский как дополнительный язык

    \setmainfont{Times New Roman} % устанавливаем главный шрифт 
    \setsansfont[Ligatures=TeX]{Arial} % устонавливаем дополнительный шрмфт 

    \newfontfamily\cyrillicfont[Script=Cyrillic]{Times New Roman} % делаем так, чтобы русский язык мог отображаться главным шрифтом 
    \newfontfamily\cyrillicfontsf[Script=Cyrillic]{Arial} % делаем так, чтобы русский язык мог отображаться дополнительным шрифтом 

    \newfontfamily\englishfont{Times New Roman} % делаем так, чтобы английский язык мог отображаться главным шрифтом 
    \newfontfamily\englishfontsf{Arial} % делаем так, чтобы английский язык мог отображаться дополнительным шрифтом 

% Рисует рамку
\usepackage{xltxtra}

\unitlength=1mm
\oddsidemargin=-0.7mm
\topmargin=-2.45cm
\def\Box#1#2{\makebox(#1,5){#2}}
\def\simpleGrad{\noindent\hbox to 0pt{
        \vbox to 0pt{
          \noindent\begin{picture}(184,286)(5.5,0.7)
            \linethickness{0.5mm}
            \put(0,0){\framebox(184,286){}}
            \end{picture}
}}}

\makeatletter
\def\@oddhead{\simpleGrad}
\def\@oddfoot{}
\makeatother

\pagenumbering{gobble}

\begin{document}

\begin{Center}{
\cyrillicfontsf 
\englishfontsf
\fontsize{16pt}{0pt}\selectfont 
  Министерство образования Республики Беларусь \vspace{20pt}

  Учреждение образования \vspace{8pt}

  <<Брестский государственный технический университет>> \vspace{20pt}

  Кафедра <<ИИТ>> \vspace{21pt}
}\end{Center}

\begin{FlushLeft} {
\cyrillicfontsf 
\englishfontsf
\fontsize{12pt}{18pt}\selectfont 

  \begin{tabular}{p{8.35cm} p{5cm}}
    & <<К защите допускаю>> Заведующий кафедрой 
  \end{tabular}
  
  \vspace{14pt}
  
  \begin{tabular}{p{8.35cm} p{8cm}}
    & \line(1, 0){3.5cm} \hspace{0.1cm} В.А. Головко \\
    & "\line(1, 0){1.0cm}" \line(1, 0){2.25cm} \hspace{0.1cm} 2025 г. \\
  \end{tabular}
  
}\end{FlushLeft}

\vspace{24pt}

\begin{Center}{
\cyrillicfontsf 
\englishfontsf
\bfseries
\fontsize{16pt}{24pt}\selectfont 
  \par РАЗРАБОТКА ПЛАТФОРМЫ ДЛЯ ОБМЕНА 

  \par ОБРАЗОВАТЕЛЬНЫМИ РЕСУРСАМИ 

  \par И ИХ ИЗУЧЕНИЯ 

  \par
}\end{Center}

\vspace{24pt}

\begin{Center}{
\cyrillicfontsf 
\englishfontsf
\fontsize{12pt}{0pt}\selectfont 
    ПОЯСНИТЕЛЬНАЯ ЗАПИСКА К ДИПЛОМНОМУ ПРОЕКТУ
}\end{Center}

\vspace{32pt}

\begin{Center}{
\cyrillicfontsf 
\englishfontsf
\bfseries
\fontsize{16pt}{0pt}\selectfont 
    БрГТУ201111-05 81 00
}\end{Center}

\vspace{22pt}

\begin{Center}{
\cyrillicfontsf 
\englishfontsf
\fontsize{16pt}{0pt}\selectfont 
    Листов 1111
}\end{Center}

\vspace{22pt}

\begin{FlushLeft} {
\cyrillicfontsf 
\englishfontsf
\fontsize{14pt}{0pt}\selectfont 

  \begin{tabular}{p{6.5cm} p{4.767cm} p{5.633cm}}
    \raggedleft Заведующий кафедрой & & В. А. Головко  
  \end{tabular}

  \vspace{14pt}
  
  \begin{tabular}{p{6.5cm} p{4.767cm} p{5.633cm}}
    \raggedleft Руководитель & & Ю. И. Давидюк \\ 
  \end{tabular}

  \vspace{14pt}

  \begin{tabular}{p{6.5cm} p{4.767cm} p{5.633cm}}
    \raggedleft Консультант & & Ю. И. Давидюк \\
    \raggedleft по основным вопроса & &
  \end{tabular}

  \vspace{14pt}

  \begin{tabular}{p{6.5cm} p{4.767cm} p{5.633cm}}
    \raggedleft Выполнил & & М. В. Седко \\
  \end{tabular}

  \vspace{14pt}

  \begin{tabular}{p{6.5cm} p{4.767cm} p{5.633cm}}
    \raggedleft Нормоконтроль & & М. В. Неседко \\
  \end{tabular}

  \vspace{14pt}

  \begin{tabular}{p{6.5cm} p{4.767cm} p{5.633cm}}
    \raggedleft Рецензент & & М. В. Неседко \\
  \end{tabular}
  
}\end{FlushLeft}

\begin{Center}{
\cyrillicfontsf 
\englishfontsf
\fontsize{14pt}{21pt}\selectfont 
    2024 г.
}\end{Center}

\end{document}
